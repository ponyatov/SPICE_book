\clearpage
\secly{Установка программного обеспечения}

\subsecly{\ngs}

\url{http://ngspice.sourceforge.net/}

\medskip\noindent
OpenSource-реализация пакета SPICE: симулятор электрических и электронных схем,
который позволяет разрабатывать разнообразные аналогово-цифровые электронные
устройства и интегральные схемы.

\subsecly{\ki}

\noindent
программный комплекс класса EDA САПР с открытым исходным кодом, предназначенный
для разработки электрических схем и печатных плат.

\subsecly{подготовка проекта}

github: \url{https://github.com/ponyatov/SPICE}

\begin{verbatim}
~$ git clone -o gh https://github.com/ponyatov/SPICE ~/SPICE
~$ cd ~/SPICE ; make install
\end{verbatim}
\begin{verbatim}
~/SPICE/$ git pull -v
~/SPICE/$ make upgrade
\end{verbatim}

\noindent
Для быстрого старта вы можете склонировать себе полный набор файлов этой книжки,
включая все файлы примеров, полный текст книги в формате \LaTeX, иллюстрации,
примеры исходного кода библиотек и т.п.

В комплекте идёт файл управления проектом \mkf\ \ref{make}, в котором есть цели
для установки и обновления нужных пакетов операционной системы (с некоторыми
излишествами).

\clearpage
Если вы не хотите пользоваться \git\ \ref{git}

% \clearpage
\subsecly{\linux}

\noindent
Предпочтительнее сразу учиться работать в \linux: есть определённые риски, что в
какой-то момент придётся переходить на отечественные процессоры, а там это
единственная доступная операционная система\note{собственно, приказ о запрете
применения импортного с 2025 года \emph{уже подписан}, и \win\ вполне может тоже
попасть в список}. Как вариант, компьютеры Raspberry Pi являются дешёвой
альтернативой полноразмерной персоналке.

\begin{verbatim}
$ sudo apt update
$ sudo apt install -yu git make curl
\end{verbatim}

\begin{description}[nosep]
\item{\textbf{sudo}} вы работаете из под пользователя, а для запуска
                    \term{пакетного менеджера} нужны привелегии администратора
\item{\textbf{apt}} в Debian GNU/Linux и Ubuntu используется \prog{apt}
\item{\textbf{-y}
}\item{\textbf{-u}
}\item{\textbf{git}}
\item{\textbf{make}}
\item{\textbf{curl}}
\end{description}

\subsecly{\win}
