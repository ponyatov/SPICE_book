\clearpage
\secly{Введение}

В связи со сложной экономической и политической ситуацией, появилась идея
создать курс факультативных занятий на базе студенческого клуба ``Контур'',
чтобы помочь желающим слезть с цифровых технологий, и освоить создание устройств
автоматики и хоббийной электроники на аналоговой элементой базе старого образца:
дискретные компоненты, ИМС малой степени интеграции, возможно вакуумные
лампы\note{аудио- и музыкальная аппаратура, радио, и техника больших токов}.

Проблема заключается в том, что отечественная промышленность не готова выпускать
аналоги сложных цифровых компонентов\note{процессоры, микроконтроллеры, ОЗУ,..}\
в объеме, достаточном для импортозамещения таких \emph{массовых}\ цифровых
платформ как Arduino и Raspberry. С остальным разнообразием комплектующих также
случилось импортозащемление, и не получится просто пойти в магазин и купить
деталей на какой-нибудь усилитель на TDA.

Вопрос о возможности внешних поставок компонентов пока под вопросом, как и их
цена в розничных количествах. Локальное производство скорее всего будет работать
только на крупных заказчиков\note{космос, военка, промышленная автоматика}, и
тратить все усилия на создание и восстановление технологий производства полного
цикла, а не на увеличения доступности элементной базы.

Главное достоинство старых дискретных технологий\ --- их \emph{быстрее}\ и
дешевле запустить в производство, и \emph{масштабировать}. Ещё остались заводы,
которые продолжают выпускать относительно простые компоненты, и использование
``дубовой'' электроники конца 80х на ближайших сроках более реально, чем мечтать
и гоняться за нанометрами.

Пока в свободном доступе были копеешные микроконтроллеры, не возникало никакого
желания возиться со схемами, которые требуют пайки и наладки вместо секундного
перепрограммирования. Теперь коробки со старой ``рассыпухой'' уже вызывают
интерес\ --- не использовать ли для очередной поделки аналоговую схему, вместо
того чтобы искать и отдавать кучу денег за микроконтроллер?

Задача этого пособия\ --- заполнить пробел между цифровой ``пердуиной'' и
профессиональной схемотехникой, и дать базовые навыки проектирования и отладки
простых аналоговых устройств.
